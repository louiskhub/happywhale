\section{Introduction}
\addcontentsline{toc}{section}{Introduction}
\label{sec:intro}

As climate change and environmental pollution intensify, nature conservation becomes increasingly important. A part of nature conservation is also the study of animal behavior, migration routes and population density, to better understand the problem and obstacles certain species are facing, so we can tackle and prevent these problems to the best of our abilities. To do this, we must be able to differentiate between individuals of a species. In humans, this is easily done by face or fingerprint recognition. But what about animals? \\
To date researchers manually differentiate them by the shape and markings on their tails, dorsal fins, heads and other body parts. This is time consuming and difficult work, since it takes the eye of a good researcher and much time to identify, match or tell individual animals apart. \\ Hence the question arises: When we can use automated identification for humans, is there a similar approach possible for animals? \\
The recent advances in facial recognition were mostly commercially motivated. Photo identification for animals seems like a less lucrative endeavour which moves it out of the research limelight. However, a technique like this could simplify the analysis of wildlife 
and therefore nature conservation significantly.\\
This is why we decided to take on the Kaggle Happywhale challenge \cite{kaggle}. This goal of this challenge is to train a machine learning model to identify whales and dolphins individuals. The competitions winning model will be used on happywhale.com \cite{happywhaledotcom}, a research collaboration platform which aims at increasing global understanding and caring for marine animals. \\ 
We want to use this challenge to look for relevantly easy approaches that could be implemented by conservation organizations to automatically detect individuals on their data-sets as well.
